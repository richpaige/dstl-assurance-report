\documentclass{llncs}

\usepackage{graphicx}
\usepackage{url}
\usepackage{courier}
\usepackage{listings}


%!TEX root = ./paper1.tex

\lstset{
  float=tb,
	captionpos=b,
	breaklines=true,
	xleftmargin=20pt,
	basicstyle=\ttfamily\scriptsize,
	numberstyle=\tiny,
	flexiblecolumns=true,
	numbers=left,
	nolol=false,
	tabsize=2
}

\lstdefinelanguage{OCL}{
morekeywords={import,if,then,else,endif,self,and,true,false,def,includes,OclElement,package,let,in},
sensitive=true,
morecomment=[l]{--},
morestring=[b]",
morestring=[b]',
showstringspaces=false
}

\lstdefinelanguage{QVTo}{
morekeywords={import,modeltype,uses,transformation,inout,in,out,configuration,property,main,var,if,then,else,endif,map,new,self,library,helper, mapping, and,return, when, where, object, true, false, result},
sensitive=true,
morecomment=[l]{--},
morecomment=[l]{//},
morestring=[b]",
morestring=[b]',
showstringspaces=false
}

\lstdefinelanguage{Acceleo}{
morekeywords={template, file, if, else, for},
sensitive=true,
morecomment=[l]{--},
morecomment=[l]{//},
morestring=[b]",
morestring=[b]',
showstringspaces=false
}

\lstdefinelanguage{MWE}{
morekeywords={module, import, var, true, false, },
sensitive=true,
morecomment=[l]{//},
morestring=[b]",
showstringspaces=false
}

\lstdefinelanguage{Java}{
morekeywords={class, private, public, true, false, new, if, for, int, return, void, extends, implements, this, null, super, import, package},
sensitive=true,
morecomment=[l]{//},
morestring=[b]",
showstringspaces=false
}

\lstdefinelanguage{JastAdd}{
morekeywords={abstract, ast, syn, inh, eq, boolean, int, false, true, if, for, return},
morestring=[b]',
sensitive=true
}

\lstdefinelanguage{NaBL}{
morekeywords={rules, defines, unique, non, refers, to},
sensitive=true
}

\lstdefinelanguage{Gra2Mol}{
morekeywords={rule, from, to, queries, mappings, skip, end_rule},
morestring=[b]',
sensitive=true
}

\lstdefinelanguage{Xtext}{
morekeywords={terminal, returns, grammar, import, fragment, current},
morestring=[b]',
morecomment=[l]{//},
sensitive=true
}

\lstdefinelanguage{Xtend}{
morekeywords={FOR, ENDFOR, IF, ELSE, ENDIF, def, protected, void, new, var, typeof, return},
morestring=[b]',
morestring=[b]",
morecomment=[s]{/*}{*/},
sensitive=true
}

\lstdefinelanguage{CS2AS}{
morekeywords={source, target, nameresolution, named, element, exports, for, from, all, children, resolution, helpers, mappings, map, disambiguation, lookup, lookupFrom, resolve, trace, when, occluding, nested,scope, def, protected, import,if,then,else,endif,self,and,true,false,def,includes,OclElement,package,let,in, void, new, var, typeof, return},
morestring=[b]',
morestring=[b]",
morecomment=[s]{/*}{*/},
sensitive=true
}



\begin{document}


\title{Technical Obsolescence Management Strategies for Safety-Related Software for Airborne Systems -- Assurance Report}

\author{Richard F. Paige\inst{1}, Simos Gerasimou\inst{1} and Dimitris Kolovos\inst{1}}

\institute{
Department of Computer Science, University of York, UK.\\
\email{[richard.paige,simos.gerasimou,dimitris.kolovos]\_at\_york.ac.uk}
}
\maketitle

\begin{abstract}
This report briefly summarises the technical challenges associated with certification (and re-certification)
of software after a particular obsolescence management strategy -- described in a companion report -- has
been carried out. In particular, it firstly considers the types of evidence that can
be produced by the management strategy. Secondly, it considers the DO-178B objectives and analyses where specific objectives
may need to be reconsidered in order to achieve re-certification. In this manner it aims to demonstrate where
re-certification effort may need to be prioritised.Finally, it briefly comments on two issues: qualification of tools that support
the management strategy, and automatic generation of certification cases, which may help to reduce the amount of effort 
required in supporting re-certification.
\end{abstract}

\section{Introduction}
Complex software systems deployed in safety-critical and business-critical 
application domains (e.g., avionics, defence, healthcare) are meant to provide 
service for decades. Although many of these systems withstand technological 
evolution and infrequently undergo substantial changes, they will likely face 
software obsolescence problems during their lifetime. 
Resolving these obsolescence problems is an expensive, time-consuming and 
labour intensive process. The costs

In this project, we explore reactive strategies for managing software 
obsolescence in safety-related software for airborne 
systems~\cite{ProjectResponse}. More specifically, we 
investigate the extent to which \textit{software modernisation} -- i.e., 
approaches that involve changes to the system 
structure, adaptation to more advanced technologies, and functionality 
enhancement -- can mitigate the problem of software obsolescence and extend the 
life, performance and reliability of existing systems. We adopt an 
experimental-based approach and use a set of demonstrators to explore different 
facets of software modernisation, 
including reverse engineering, program understanding, demonstration of 
functional equivalence of 
migrated code, change in hardware platform, maintenance of performance, and 
preservation of Design Assurance Levels (DAL).

In this report we focus on certification and assurance issues, specifically: if the obsolescence management strategy
presented in the companion report is applied to a system at DAL-C, what are the certification issues associated with
the migrated software? In particular, what effort will be required to re-certify the software system, and what evidence
can the strategy provide in order to support re-certification?

The emphasis in this 11-month project sponsored by DSTL has been on building and evaluating a technical solution to
software obsolescence on several case studies; assurance and certification has been a secondary concern. As such, this
report is primarily speculative. It is structured as follows. Firstly, we describe the evidence that is produced by the 
obsolescence strategy and technical solution that is relevant to certification and recertification. We also briefly describe
the toolchain that has been developed to support the obsolescence strategy, so that the report is moderately self-contained
and hence easier to follow. Next, we analyse the DO-178B objectives and summarise which specific objectives \textit{must} and
\textit{could} be reconsidered as a result of applying the strategy. Finally, we comment on two issues: that of tool qualification
(and the challenges that will arise in qualifying the toolchain developed within this project), and of automatically generating 
assurance/safety cases from the engineering artefacts that are provided as input to the obsolescence strategy.

\section{Evidence and Assumptions}

\section{Certification and DO-178B Objectives}

\section{The Future: Tool Qualification and Assurance Case Generation}

\section{Conclusions}

\bibliographystyle{abbrv}
\bibliography{bibliography}
\renewcommand{\baselinestretch}{1.0}


\end{document}